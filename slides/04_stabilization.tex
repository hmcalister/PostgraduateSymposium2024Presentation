\section{Stabilizing the Network}

% --------------------------------------------------------------------------------

% \begin{frame}
% \frametitle{Stabilizing the Network}

% \begin{itemize}
%     \item The optimal hyperparameter region shrinks as \(n\) grows.
%     \item The optimal region also shifts considerably as \(n\) grows.
%     \item When \(n\) reaches some critical threshold, the optimal region is disrupted and the network does not learn at all.
% \end{itemize}

% \end{frame}

%------------------------------------------------

\begin{frame}
    \frametitle{Network Dynamics Step by Step}
    \begin{gather*}
        \tanh \left[ \beta \sum_{\mu} \left( f_n \left(\zeta_{\mu, i} + \sum_{j \neq i} \zeta_{\mu, j} \xi_{j} \right) - f_n \left( -\zeta_{\mu, i} + \sum_{j \neq i} \zeta_{\mu, j} \xi_{j} \right) \right) \right]
    \end{gather*}

    \begin{enumerate}
        \only<1,2>{\item Calculate similarities \(\zeta \cdot \xi_{+1}, \; \zeta \cdot \xi_{-1}\)}
        \only<3->{\item \textbf{Calculate similarities \(\zeta \cdot \xi_{+1}, \; \zeta \cdot \xi_{-1}\)}}
        \only<1>{\item Pass similarities through interaction function \(f_n\)}
        \only<2->{\item \textbf{Pass similarities through interaction function \(f_n\)}}
        \item Sum the result over all memories \(\sum_\mu\)
        \item Multiply by a scaling factor \(\beta\)
        \item Pass through activation function (e.g. Sign or tanh)
    \end{enumerate}

    \only<2>{
        \begin{align*}
            f_n\left(x\right) = x^n
        \end{align*}
    }

    \only<3>{
        \begin{align*}
            \zeta, \xi \in [-1,1]^N &\implies \zeta \cdot \xi \in [-N, N] \\
        \end{align*}
    }

    \only<4>{
        \begin{align*}
            \zeta, \xi \in [-1,1]^N &\implies \zeta \cdot \xi \in [-N, N] \\
            f_n\left(\zeta \cdot \xi\right) &= \left(\zeta \cdot \xi\right)^n 
        \end{align*}
    }

    \only<5>{
        \begin{align*}
            \zeta, \xi \in [-1,1]^N &\implies \zeta \cdot \xi \in [-N, N] \\
            f_n\left(\zeta \cdot \xi\right) &= \left(\zeta \cdot \xi\right)^n \\
                &= N^n
        \end{align*}
    }

\end{frame}

\begin{frame}
    \frametitle{How large is too large?}

    \begin{center}
        \begin{tabular}{| c | c |}
            \hline
            Network parameters & Interaction function value \\
            \hline
            \hline
            \(N=100, n=2\) & \(10^{4}\) \\
            \hline
            \(N=100, n=5\) & \(10^{10}\) \\
            \hline
            \(N=100, n=10\) & \(10^{18.89}\) \\
            \hline
            \(N=100, n=20\) & \(10^{40}\) \\
            \hline
        \end{tabular}
    \end{center}

    \pause
    Maximum value of a float32 is \(\approx 3.4 \cdot 10^{38}\). Default data type of PyTorch.

    \pause
    Maximum value of a float64 is \(\approx 1.8 \cdot 10^{304}\). Default data type of NumPy.
\end{frame}

\begin{frame}
    \frametitle{Stabilizing the Network}
    
    \begin{itemize}
        \item When \(n\) reaches some critical threshold, the optimal region is disrupted and the network does not learn at all.
        % \pause
        % \begin{itemize}
        %     \item Large \(n\) makes prototype memories.
        %     \item Prototype memories have high similarities.
        %     \item High similarities and large \(n\) causes overflow.
        % \end{itemize}
        \pause
        \item The optimal hyperparameter region shrinks as \(n\) grows.
        \pause
        \item The optimal region also shifts considerably as \(n\) grows.
    \end{itemize}    
\end{frame}
    

\begin{frame}
    \frametitle{Fixing the Problem}
    \only<1>{
        \begin{gather*}
            \tanh \left[ \beta \sum_{\mu} \left( f_n \left(\zeta_{\mu, i} + \sum_{j \neq i} \zeta_{\mu, j} \xi_{j} \right) - f_n \left( -\zeta_{\mu, i} + \sum_{j \neq i} \zeta_{\mu, j} \xi_{j} \right) \right) \right]
        \end{gather*}
    }
    \only<2>{
        \begin{gather*}
            \tanh \left[ \textcolor{red}{\boldsymbol \beta} \sum_{\mu} \left( f_n \left(\zeta_{\mu, i} + \sum_{j \neq i} \zeta_{\mu, j} \xi_{j} \right) - f_n \left( -\zeta_{\mu, i} + \sum_{j \neq i} \zeta_{\mu, j} \xi_{j} \right) \right) \right]
        \end{gather*}
    }
    \only<3>{
        \begin{gather*}
            \tanh \left[ \sum_{\mu} \left( \textcolor{red}{\boldsymbol \beta} f_n \left(\zeta_{\mu, i} + \sum_{j \neq i} \zeta_{\mu, j} \xi_{j} \right) - \textcolor{red}{\boldsymbol \beta} f_n \left( -\zeta_{\mu, i} + \sum_{j \neq i} \zeta_{\mu, j} \xi_{j} \right) \right) \right]
        \end{gather*}
    }
    \only<4>{
        \begin{gather*}
            \tanh \left[ \sum_{\mu} \left( f_n \left(\textcolor{red}{\boldsymbol \beta} \left(\zeta_{\mu, i} + \sum_{j \neq i} \zeta_{\mu, j} \xi_{j} \right) \right) - f_n \left(\textcolor{red}{\boldsymbol \beta} \left( -\zeta_{\mu, i} + \sum_{j \neq i} \zeta_{\mu, j} \xi_{j} \right) \right) \right) \right]
        \end{gather*}
        \begin{center}
            If \(f_n\) is homogenous: \(f(\alpha x) = \alpha^k f(x)\). \\
            Weaker than linear -- includes Polynomial and Rectified Polynomial.
        \end{center}
    }

    \only<5>{
        \begin{gather*}
            \tanh \left[ \sum_{\mu} \left( f_n \left(\textcolor{red}{\frac{\boldsymbol \beta}{N}} \left(\zeta_{\mu, i} + \sum_{j \neq i} \zeta_{\mu, j} \xi_{j} \right) \right) - f_n \left(\textcolor{red}{\frac{\boldsymbol \beta}{N}} \left( -\zeta_{\mu, i} + \sum_{j \neq i} \zeta_{\mu, j} \xi_{j} \right) \right) \right) \right]
        \end{gather*}
        \begin{center}
            If \(f_n\) is homogenous: \(f(\alpha x) = \alpha^k f(x)\). \\
            Weaker than linear -- includes Polynomial and Rectified Polynomial.
        \end{center}
    }
\end{frame}
